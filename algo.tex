\documentclass{article}

%\usepackage{epsfig}

%\usepackage{algorithm2e}
%\usepackage[linesnumbered]{algorithm2e}
%\usepackage[linesnumbered,ruled]{algorithm2e}
\usepackage[linesnumbered,ruled,vlined]{algorithm2e}

\title{Merge Sort}
%\vspace{-8ex}}
  %\date{}

\begin{document}
\maketitle

Algorithm~\ref{algo:mergesort} is a recursive algorithm to sort an input array. It uses the Algorithm~\ref{algo:merge} as a subroutine abcd efgh. 
Algorithm~\ref{algo:merge} merges two sorted arrays and outputs the merged array in sorted order.


\begin{algorithm}
\DontPrintSemicolon
\KwIn{An array $A$ and two indices $i, j$ between which the sorting is to be done}
\KwOut{$A$ with its entries between indices $i$ and $j$ sorted}
\If{$i < j$}{
	$mid \gets (i+j)/2$\;
	Mergesort($A,i,mid$)\;
	Mergesort($A,mid+1,j$)\;
	Create temporarily $C[0.... j-i]$\;
	Merge$(A[i.... mid],A[mid+1.... j],C)$\;
	Copy $C[0....j-i] to A[i.... j]$\;
}
\Return{$A$}\;
\caption{The \textbf{Mergesort} Algorithm}
\label{algo:mergesort}
\end{algorithm}

\begin{algorithm}
\DontPrintSemicolon
\KwIn{Three arrays $A[0.... n-1], B[0.... m-1],C$ where $A$ and $B$ are in sorted order}
\KwOut{Array $C$ containing all the entries of $A$ and $B$ in sorted order}
$i \gets 0$\;
$j \gets 0$\;
$k \gets 0$\;
\While{$i < n \land j < m$}{
	\If{$A[i] < B[j]$}{
		$C[k] \gets A[i]$\;
		$k \gets k+1$\;
		$i \gets i+1$\;
	}
	\Else{
		$C[k] \gets B[j]$\;
		$k \gets k+1$\;
		$j \gets j+1$\;
	}
}
\While{$i < n$}{
	$C[k] \gets A[i]$\;
	$k \gets k+1$\;
	$i \gets i+1$\;		
}
\While{$j < m$}{
	$C[k] \gets B[j]$\;
	$k \gets k+1$\;
	$j \gets j+1$\;		
}		
\Return{$C$}\;
\caption{The \textbf{Merge} procedure. Refer to \cite{atharvaveda} for more details.}
\label{algo:merge}
\end{algorithm}

\bibliographystyle{abbrv}
\bibliography{refs}

\end{document}